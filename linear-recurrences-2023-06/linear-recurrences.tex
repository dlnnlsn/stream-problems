\documentclass{article}

\usepackage{amsmath}
\usepackage{amssymb}

\title{Problems Involving Linear Recurrence Relations}
\author{Dylan Nelson}
\date{June 2023}

\begin{document}
\maketitle

\begin{enumerate}

\item {\textbf{PAMO 2017 Problem 1}}

We consider the real sequence $(x_n)$ defined by $x_0 = 0$, $x_1 = 1$, and $x_{n + 2} = 3x_{n + 1} - 2x_n$ for $n = 0, 1, 2, \dots$

We define the sequence $(y_n)$ by $y_n = {x_n}^2 + 2^{n + 2}$ for every non-negative integer $n$.

Prove that for every $n > 0$, $y_n$ is the square of an odd integer.

\item {\textbf{PAMO 2019 Problem 1}}

Let $(a_n)_{n = 0}^{\infty}$ be a sequence of real numbers defined as follows:
\begin{itemize}
    \item $a_0 = 3$, $a_1 = 2$, and $a_2 = 12$; and
    \item $2a_{n + 3} - a_{n + 2} - 8a_{n + 1} + 4a_n = 0$ for $n \geq 0$.
\end{itemize}

Show that $a_n$ is always a strictly positive integer.

\item {\textbf{{\itshape Not} PAMO 2023 Problem 3}}

Consider a sequence of real numbers defined by:
\begin{align*}
    x_1 & = c \\
    x_{n + 1} & = cx_n + \sqrt{c^2 - 1} \sqrt{x_n^2 - 1} \text{ for all } n \geq 1.
\end{align*}

Find a closed form for $x_n$.

\item {\textbf{BMO 2012 Round 2 Problem 4}}

Show that there is a positive integer $k$ with the following property: if $a$, $b$, $c$, $d$, $e$, and $f$ are integers, and $m$ is a divisor of
\[
    a^n + b^n + c^n - d^n - e^n - f^n
\]
for all integers $n$ in the range $1 \leq n \leq k$, then $m$ is a divisor of $a^n + b^n + c^n - d^n - e^n - f^n$ for all positive integers $n$.

\item

Suppose that the sequence $(T_n)$ satisfies the recurrence relation
\[
    T_{n + 2} + aT_{n + 1} + bT_n = 0
\]
for all $n \geq 0$. Show that
\[
    T_{n + 2} T_n - T_{n + 1}^2 = b^n \left( T_2 T_0 - T_1^2 \right)
\]
for all $n \geq 0$.

\item {\textbf{Korea 2019 Final Round Problem 3}}

Prove that there exist infinitely many positive integers $k$ such that the sequence ${x_n}$ satisfying $x_1 = 1$, $x_2 = k + 2$, and
\[
    x_{n + 2} - (k + 1) x_{n + 1} + x_n = 0 
\]
for all $n \geq 0$ does not contain any prime number.

\item

Find all functions $f : \mathbb{N}_0 \to \mathbb{N}_0$ such that
\[
    f(f(n)) + f(n) = 6n
\]
for all $n \in \mathbb{N}_0$.

\item

Find a closed form for the sequence $T_n$ satisfying $T_0 = 0$ and
\[
    T_{n + 1} = 2T_n + 2^{n + 1} - 2
\]
for all integers $n \geq 0$.

\item

Let $A$ and $B$ be two fixed real numbers. Find all sequences $(T_n)$ such that
\[
    T_{n + 2} - T_{n + 1} - 2T_n = \begin{cases}
        A & \text{ if } n \equiv 0 \pmod 2 \\
        B & \text{ if } n \equiv 1 \pmod 2.
    \end{cases}
\]

\item

Find all sequences $T_n$ such that
\[
    T_{n + 2} + T_{n + 1} + T_n = 3 \left( n \bmod 3 \right)
\]
for all $n \geq 0$. (Here $n \bmod 3$ is the remainder when $n$ is divided by $3$)

\item

Find all sequences $T_n$ such that
\[
    T_{n + 2} - 3T_{n + 1} + 2T_n = n \left( n \bmod 4 \right)
\]
for all $n \geq 0$.

\item

Show that the $n^{\text{th}}$ Fibonacci number $F_n$ is equal to the closest integer to
\[
    \frac{1}{\sqrt{5}} \left( \frac{1 + \sqrt{5}}{2} \right)^n.
\]

\end{enumerate}

\end{document}
